\documentclass[a4paper]{article}

\usepackage[english]{babel}
\usepackage[UKenglish]{isodate}
\usepackage[parfill]{parskip}
\usepackage{float}
\usepackage{graphicx}
\usepackage{booktabs}
\usepackage{csquotes}
\usepackage{listings}
\usepackage[margin=2.5cm]{geometry}
\usepackage{enumerate}
\usepackage{siunitx}
\usepackage{appendix}

\newcommand{\nth}{\textsuperscript}

\title{Assignment 2 Report}
\author{Kevin Zihong Ni\\\texttt{z5025098}}

\begin{document}

\maketitle

My implementation of the link state routing protocol was written in C++ (as authorised by the lecturer Apro S. Kanhere
over email). I chose to do this as C++, unlike C offers the standard template library, allowing the use of data structures such as
maps and priority queues which would have required significant amounts of time investment to implement on my own. The code does 
\textbf{not} require a C++11 compiler and works with the version of g++ installed on the CSE SSH servers. I have also included
a makefile to aid in compilation.

\section{Design Overview}

The algorithm's initialisation only consists of setting up the socket and reading from the configuration file. The details of 
the node's neighbour's (the link state) are stored in a "packet" data structure which can be easily be broadcasted when needed.


The main loop of the algorithm can be broken up into 4 parts, mediated by a global timer: 
\begin{enumerate}
	\item Check the global timer against the last time this part was executed and send a packet containing the node's link state
		to all of its neighbours when that value is greater than 1 second. 
	\item Check the global timer against the last time this part was executed and perform a Dijkstra's Algorithm search on
		the current graph.
	\item Listen for new incoming packets, update the graph if needed and retransmit the packet to all neighbours except
		for the one that the packet came from. Also keep a record of the last time a packet was 
		received from a particular neighbour.
	\item Look through the list of neighbours. If the last time one was heard is greater than 3 seconds, presume that node
		had died.
\end{enumerate}

\section{Data Structures}

\subsection{Packets}

The C++ definition of the data structure I used to send link states between nodes is as follows:
\begin{lstlisting}[language=C++]
struct Header{
	unsigned char id;
	unsigned char seq;
	int len;
};

struct Packet{
	Header header;
	std::pair<unsigned char, double> data[MAX_NODES];
};
\end{lstlisting}

\verb|data| is an array of every neighbour of the node that originally created the packet and their cost.
\verb|id| refers to the ID of the node that this packet originated from (\textbf{not} the node that last retransmitted the
packet). This means that there is a link between \verb|id| and every node listed in \verb|data| (which also has its corresponding
cost). \verb|seq| is a sequence number for the packet (used to prevent unnecessary retransmissions) and \verb|len| is the length 
of the array \verb|data|.

\subsection{Network Topology}
The network topology was represented using a $26 \times 26$ adjacency matrix. The index of a node was simply the ASCII value of
its designated letter minus the ASCII value of 'A'. Since there are only 26 letters, 26 was an acceptable size for the array.
This data structure was chosen over others (such as adjacency lists) due to its relative simplicity and the small upper bound on 
the size of the sample data.

\section{Node Failures}
Nodes keep track of the last time they have "heard" from each of its neighbours. If this value is ever more than 3 seconds behind
the global clock (i.e. 3 missed packets), the corresponding neighbour is then presumed "dead" and measures can be taken to remove
them from the graph. First, the cost of the link to that node is set to infinity in the link state packet. This is done instead
of outright removing the entry since the other nodes need to be informed of the change in cost. The next time the node's link
state is broadcast, the other nodes will automatically overwrite the existing costs of this edge with infinity on their own graphs
upon receiving the packet. Second, the cost of every edge going to the dead node must also be set to infinity in the graph as the 
automatic mechanism only applies to nodes receiving the packet. Lastly, the dead node must be removed from the list of neighbours
so that network resources won't be wasted on attempting to contact a dead node.

\section{Restricting Excessive Link State Broadcasts}
Nodes keep an array of sequence numbers for each possible letter (even if they are not in the network). These sequence numbers
correspond with the sequence numbers present on each packet. When a packet is received, its sequence number is checked against
the recorded sequence number. If the number is "ahead" of the last recorded sequence number, then that packet is retransmitted
to all of the node's neighbours except for the one that the packet was received from. If the sequence number is equal to or behind
the recorded one, then the packet is discarded. Integer overflows can be accounted for by subtracting the two numbers and
comparing the result against a constant.

\end{document}
